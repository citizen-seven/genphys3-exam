\documentclass{letask}

\begin{document}
\include{cover}

\paragraph{Цель работы:} Изучение характера связанных колебаний магнитных стрелок двух расположенных рядом компасов.

\paragraph{В работе используются:} неокуб, линейка, штатив, нитки, секундомер.

\section{Описание установки}  

Две стрелки, собранные из шести магнитных шариков неокуба, подвесим на нити за середину на некотором расстоянии $l$ друг от друга так, чтобы их оси совпадали. Под осью понимается прямая соединяющая Северный и Южный конец стрелки. Стрелки будут направлены по магнитному полю земли. Отклоним одну из стрелок из положения равновесия, придерживая вторую. Будем наблюдать за колебаниями подобной системы. 

Если в начальный момент времени ($t = 0$) мы отклоним первую стрелку, вторую при этом придерживая в состоянии равновесия, а затем одновременно отпустим обе стрелки, мы будем наблюдать уменьшение амплитуды колебаний первой стрелки, в то время как для второй стрелки угол отклонения от оси будет расти. В некоторый момент первая стрелка остановится, при этом вторая стрелка будет иметь амплитуду и энергию колебания первой стрелки в начальный момент. Из-за наличия сил трения колебания будут постепенно затухать, но заметно это становится после 4-5 колебаний, следовательно трение пренебрежимо мало.

Подобная перекачка энергии от одной стрелки к другой называется биением. Система ведет себя как связанные маятники, но в данном случае в роли соединения выступает магнитное взаимодействие между двумя намагниченными стрелками. 

Если отклонить стрелки одновременно в одном направлении, будем наблюдать так называемые синфазные колебания. 
Если отклонить стрелки в разных направлениях одновременно, будем наблюдать противофазные колебания. 

Для объяснения природы таких колебания слегка изменим условия эксперимента: поместим стрелки так, чтобы их оси были параллельны. Так биения видно еще лучше, в связи с природой распределения магнитного поля стрелки. На конце стрелки поле более неоднородно, чем на перпендикуляре к оси стрелки в плоскости стрелки.


\section{Теория}

\subsection{Уравнение движения}  

\begin{equation}
\mathfrak{I} \cdot \vec{\varepsilon} = \vec{M},
\end{equation}
где $\mathfrak{I}$ - момент инерции стрелки относительно центра масс, $M$ - момент внешних сил.

\begin{equation}
M = \left[ \vec{p_m}, \vec{B} \right],
\end{equation}
где $\vec{p_m} = \vec{I} \cdot V$ - магнитный момент, который зависит от намагниченности $I$ и объема $V$ стрелки, $B$ - вектор магнитной индукции внешнего поля (горизонтальная компонента магнитного поля Земли)

Для малых углов $\varphi$ отклонения:

\begin{equation}
M = - p_m \cdot B \cdot \sin \varphi \approx - p_m \cdot B \cdot \varphi
\end{equation}

Дополнительный момент сил:
\begin{equation}
M_1 = p_m \cdot B_1 \cdot (\varphi_2 - \varphi_1),
\end{equation}
где $\vec{B_1}$ - вектор магнитной индукции, созданной второй стрелкой.

Так как стрелки одинаковы, то $p_m$ и $\mathfrak{I}$ для них будем считать равными. При таких условиях запишем уравнения движения обеих стрелок.

\begin{equation}
\begin{gathered}
\mathfrak{I} \ddot{\varphi_1} + p_m B \varphi_1 - p_m B_1 (\varphi_2 - \varphi_1) = 0, \\
\mathfrak{I} \ddot{\varphi_2} + p_m B \varphi_2 - p_m B_1 (\varphi_1 - \varphi_2) = 0.
\end{gathered}
\end{equation}


\section{Вывод}  

\begin{itemize}
\item При определенных расстояниях между магнитными стрелками их можно рассматривать как связанные маятники и наблюдать явление биения, $\omega_b \sim \dfrac{1}{r^2}$.

\item На очень близких расстояниях не выполняется условие слабой связи и биения отсутствуют. Но из-за сильной неоднородности поля на концах стрелки мы видим нелинейные колебания.
\end{itemize}

\end{document}
