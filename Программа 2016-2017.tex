\documentclass[9pt,a4paper]{article}
\usepackage[utf8]{inputenc}
\usepackage[russian]{babel}
\usepackage[OT1]{fontenc}
\usepackage{amsmath}
\usepackage{amsfonts}
\usepackage{amssymb}
\usepackage{graphicx}
\usepackage[left=5 mm,right=5mm,top=1cm,bottom=1cm]{geometry}
\begin{document}
\begin{center}
\subsection*{ВОПРОСЫ ЭКЗАМЕНАЦИОННЫХ БИЛЕТОВ \\ «ЭЛЕКТРИЧЕСТВО И МАГНЕТИЗМ» 2015-2016 уч. г.}
\end{center}
\begin{enumerate}
\item Закон Кулона. Напряжённость электрического поля. Элементарный заряд. Принцип суперпозиции. Единицы измерения заряда (в системе Гаусса и СИ). Поле точечного диполя.
\item Теорема Гаусса для электрического поля в вакууме (интегральная и дифференциальная формы). Примеры применения. 
\item Потенциальный характер электростатического поля. Потенциал и разность потенциалов. Теорема о циркуляции в электростатическом поле. Связь потенциала с напряжённостью поля. Потенциал поля точечного диполя.  
\item Потенциал электростатического поля. Уравнение Пуассона и уравнение Лапласа. Граничные условия и метод зеркальных изображений. 
\item Проводники в электростатическом поле. Электростатическая защита. Граничные условия на поверхности проводника. Проводящий шар в электростатическом поле.  
\item Диэлектрики в электростатическом поле. Механизм поляризации диэлектриков. Свободные и связанные (поляризационные) заряды. Вектор поляризации. Связь вектора поляризации с поляризационными зарядами. Поверхностный и объёмный поляризационные заряды.  
\item Вектор поляризации и его связь с поляризационными зарядами. Вектор электрической индукции. Поляризуемость и диэлектрическая проницаемость. 
\item Теорема Гаусса для электрического поля в диэлектриках (интегральная и дифференциальная формы). Граничные условия на границе раздела двух диэлектриков.  
\item Электрическая ёмкость уединённых проводников и конденсаторов. Расчёт ёмкости плоского, сферического и цилиндрического
конденсаторов.
\item Электрическая энергия и её локализация в пространстве. Объёмная плотность энергии. Энергия диполя во внешнем поле (жёсткий и
упругий диполи). Взаимная энергия зарядов.
\item Силы, действующие на диполь в неоднородном электрическом поле. Энергетический метод вычисления сил (случаи: q=const U=const).
\item Постоянный ток. Сила и плотность тока. Сторонние силы. Закон сохранения заряда и уравнение непрерывности. Токи в неограниченных
средах.
\item Закон Ома (интегральная и локальная формы). Постоянный ток в замкнутом контуре. Электродвижущая сила. Правила Кирхгофа.
Примеры применения.
\item Работа и мощность постоянного тока. Закон Джоуля–Ленца в интегральной и локальной форме.
\item Магнитное поле постоянного тока. Вектор магнитной индукции. Сила Лоренца и сила Ампера. Закон Био–Савара. Магнитный момент
рамки с током. Момент сил, действующий на рамку с током в магнитном поле.
\item Теорема о циркуляции магнитного поля в вакууме (интегральная и дифференциальная формы). Примеры применения. Магнитное поле соленоида. Теорема Гаусса для магнитного поля (интегральная и дифференциальная формы).
\item Магнитное поле в веществе. Молекулярные токи. Вектор намагниченности и его связь с молекулярными токами (интегральная и дифференциальная формы).
\item Теорема о циркуляции магнитного поля в веществе. Вектор  $\mathbf{H}$. Применение к расчёту магнитных цепей.
\item Граничные условия для векторов $\mathbf{B}$ и  $\mathbf{H}$ на границе раздела двух магнетиков.
\item Постоянный магнит. Магнитные поля  $\mathbf{B}$ и  $\mathbf{H}$ постоянного магнита.
\item Магнитные свойства сверхпроводника I-го рода. Эффект Мейснера. Граничные условия на поверхности сверхпроводника.
Сверхпроводящий шар в магнитном поле.
\item Работа сил Ампера по перемещению витка с током в магнитном поле.
\item Электромагнитная индукция в движущихся проводниках. Правило Ленца.
\item Электромагнитная индукция в неподвижных проводниках. Фарадеевская и максвелловская трактовка явления электромагнитной индукции. Вихревое электрическое поле.
\item Нерелятивистское преобразование полей  $\mathbf{B}$ и  $\mathbf{E}$ при переходе от одной инерциальной системы к другой. Магнитное поле равномерно
движущегося заряда.
\item Движение заряженных частиц в электрическом и магнитном полях. Дрейфовое движение. Циклотронная частота.
\item Магнитный поток. Коэффициенты самоиндукции и взаимоиндукции. Индуктивность соленоида и тороидальной катушки.
\item Установление тока в цепи, содержащей индуктивность. Магнитная энергия тока. Локализация магнитной энергии в пространстве.
\item Взаимная энергия токов. Теорема взаимности. Взаимная индуктивность двух катушек на общем магнитопроводе.
\item Энергетический метод вычисления сил в магнитном поле. Подъёмная сила электромагнита.
\item Переменное электрическое поле и его магнитное действие. Ток смещения. Примеры расчёта.
\item Системы уравнений Максвелла в интегральной форме. Граничные условия. Материальные уравнения.
\item Система уравнений Максвелла в дифференциальной форме. Граничные условия. Материальные уравнения.
\item Волновое уравнение как следствие уравнений Максвелла. Плоские электромагнитные волны в однородной среде. Скорость
распространения. Поперечность электромагнитных волн. Связь полей  $\mathbf{B}$ и  $\mathbf{E}$ в плоской электромагнитной волне.
\item Монохроматическая (гармоническая) плоская волна. Стоячие электромагнитные волны. Отражение электромагнитной волны от плоской
поверхности идеального проводника.
\item Электромагнитные волны в волноводах. Простейшие типы электромагнитных волн в волноводе прямоугольного сечения. Критическая
частота. Длина волны и фазовая скорость волн в волноводе.
\item Двухпроводная линия как пример неквазистационарной цепи. Электромагнитная волна в двухпроводной линии. Скорость волны.
Волновое сопротивление. Согласованная нагрузка.
\item Поток энергии. Вектор Пойнтинга. Теорема Пойнтинга. Примеры применения.
\item Давление излучения. Опыты Лебедева. Электромагнитный импульс.
\item Излучение электромагнитных волн. Излучение колеблющегося диполя (без вывода). Диаграмма излучения. Зависимость мощности
излучения от частоты (закон Релея).
\item Отражение и преломление электромагнитных волн на плоской границе двух диэлектриков. Формулы Френеля. Коэффициенты отражения
и прозрачности. Угол Брюстера. Полное внутреннее отражение. Понятие о неоднородных волнах.
\item Скин-эффект. Толщина скин-слоя, её зависимость от частоты и проводимости.
\item Квазистационарные процессы. Уравнение гармонического осциллятора. Свободные колебания осциллятора с затуханием.
\item Коэффициент затухания, логарифмический декремент, добротность колебательного контура. Превращения энергии при затухающих
колебаниях. Энергетический смысл добротности.
\item Вынужденные колебания в линейных системах (гармоническая внешняя ЭДС). Амплитудно-фазовая характеристика линейных фильтров.
Колебательный контур. Резонанс. Ширина резонансной кривой и её связь с добротностью.
\item Процессы установления вынужденных колебаний. Биения.
\item Расчёт цепей, содержащих сопротивления, индуктивности и ёмкости при гармоническом внешнем воздействии. Метод комплексных
амплитуд. Векторные диаграммы. Резонанс.
\item Правила Кирхгофа для переменных токов. Работа и мощность переменного тока.
\item Вынужденные колебания в линейных системах под действием негармонической внешней силы – спектральный анализ линейных систем.
\item Модулированные колебания. Амплитудная и фазовая модуляция. Векторное изображение модулированных колебаний. Спектры
колебаний, модулированных по амплитуде и фазе (при синусоидальной модуляции).
\item Представление модулированных сигналов в виде суперпозиции гармонических колебаний. Опыты Мандельштама. Понятие о разложении
Фурье (ряд Фурье, интеграл Фурье). Примеры спектральных разложений. Соотношение неопределённостей.
\item Параметрические колебания. Условия возбуждения индуктивной параметрической машины, параметрический резонанс.
\item Понятие об автоколебаниях. Обратная связь. Условие самовозбуждения.
\item Понятие о плазме. Дебаевский радиус. Плазменные колебания и плазменная частота.
\item Диэлектрическая проницаемость плазмы.
\end{enumerate}
\end{document}